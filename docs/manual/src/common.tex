\begin{titlepage}
  \begin{center}

  {\Huge FT245\_SYNC\_TO\_AXIS}

  \vspace{25mm}

  \includegraphics[width=0.90\textwidth,height=\textheight,keepaspectratio]{img/AFRL.png}

  \vspace{25mm}

  \today

  \vspace{15mm}

  {\Large Jay Convertino}

  \end{center}
\end{titlepage}

\tableofcontents

\newpage

\section{Usage}

\subsection{Introduction}

\par
FT245 sync to AXIS core converts FT245 to the AXIS interface. This is done using asynchronous conversions with a few registers to sync signals in time.

\subsection{Dependencies}

\par
The following are the dependencies of the cores.

\begin{itemize}
  \item fusesoc 2.X
  \item iverilog (simulation)
  \item cocotb (simulation)
\end{itemize}

\subsubsection{fusesoc\_info Depenecies}
\begin{itemize}
\item dep\_tb
	\begin{itemize}
	\item AFRL:simulation:axis\_stimulator
	\item AFRL:utility:sim\_helper
	\end{itemize}
\end{itemize}


\subsection{In a Project}
\par
Connect the device to your AXIS bus. Connect the FT245 bus to the FTDI device.

\section{Architecture}
\par
This core is made up of a single module.
\begin{itemize}
  \item \textbf{ft245\_sync\_to\_axis} Interface AXIS to F245 device (see core for documentation).
\end{itemize}

\par
This core has 1 always blocks that are sensitive to the positive clock edge.

\begin{itemize}
\item \textbf{register signals} registers the ft245 data read fx signal and based upon its state the the state of tread outputs AXIS data.
\end{itemize}

Please see \ref{Module Documentation} for information on how the asynchronous assignments are done.

\section{Building}

\par
The FT245 sync to AXIS is written in Verilog 2001. It should synthesize in any modern FPGA software. The core comes as a fusesoc packaged core and can be included in any other core. Be sure to make sure you have meet the dependencies listed in the previous section. Linting is performed by verible using the lint target.

\subsection{fusesoc}
\par
Fusesoc is a system for building FPGA software without relying on the internal project management of the tool. Avoiding vendor lock in to Vivado or Quartus.
These cores, when included in a project, can be easily integrated and targets created based upon the end developer needs. The core by itself is not a part of
a system and should be integrated into a fusesoc based system. Simulations are setup to use fusesoc and are a part of its targets.

\subsection{Source Files}

\subsubsection{fusesoc\_info File List}
\begin{itemize}
\item src
	\begin{itemize}
	\item src/ft245\_sync\_to\_axis.v
	\end{itemize}
\item tb
	\begin{itemize}
	\item {'tb/tb\_axis.v': {'file\_type': 'verilogSource'}}
	\item {'tb/in.bin': {'file\_type': 'user', 'copyto': 'in.bin'}}
	\end{itemize}
\end{itemize}


\subsection{Targets}

\subsubsection{fusesoc\_info Targets}
\begin{itemize}
\item default
	\begin{itemize}
	\item[$\space$] Info: Default for IP intergration.
	\end{itemize}
\item lint
	\begin{itemize}
	\item[$\space$] Info: Lint with Verible
	\end{itemize}
\item sim
	\begin{itemize}
	\item[$\space$] Info: Default simulation using icarus.
	\end{itemize}
\end{itemize}


\subsection{Directory Guide}

\par
Below highlights important folders from the root of the directory.

\begin{enumerate}
  \item \textbf{docs} Contains all documentation related to this project.
    \begin{itemize}
      \item \textbf{manual} Contains user manual and github page that are generated from the latex sources.
    \end{itemize}
  \item \textbf{src} Contains source files for the core
  \item \textbf{tb} Contains test bench files for iverilog and cocotb
    \begin{itemize}
      \item \textbf{cocotb} testbench files
    \end{itemize}
\end{enumerate}

\newpage

\section{Simulation}
\par
There are a few different simulations that can be run for this core.

\subsection{iverilog}
\par
iverilog is used for simple test benches for quick verification, visually, of the core.

\subsection{cocotb}
\par
Future simulations will use cocotb. This feature is not yet implemented.

\newpage

\section{Module Documentation} \label{Module Documentation}

\begin{itemize}
\item \textbf{ft245\_sync\_to\_axis} Interfaces AXIS to the FT245.\\
\end{itemize}
The next sections document the module in great detail.

